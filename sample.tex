% 서울대학교 전기공학부 (전기컴퓨터공학부) 석사ㅡ 박사 학위논문
% LaTeX 양식 샘플
\RequirePackage{fix-cm} % documentclass 이전에 넣는다.
% oneside : 단면 인쇄용
% twoside : 양면 인쇄용
% ko : 국문 논문 작성
% master : 석사
% phd : 박사
% openright : 챕터가 홀수쪽에서 시작
\documentclass[oneside,ko,phd]{snuthesis}

%%%%%%%%%%%%%%%%%%%%%%%%%%%%%%%%%%%%%%%%%%%%%%%%%%%%%%%%%%%%%%%%%%%%%%%%%%%%%%
%%
%% Author: zeta709 (zeta709@gmail.com)
%% Releasedate: 2017/07/20
%%
%% 목차 양식을 변경하는 코드
%% * subfigure (subfig) package 사용 여부에 따라
%%   tocloft의 옵션을 다르게 지정해야 한다.
%% * Chapter 번호가 두 자리 수를 넘어가는 경우 다음과 같이
%%   필요한 만큼 "9"를 추가하면 된다.
%%   \settowidth{\mytmplen}{\bfseries\cftchappresnum9\cftchapaftersnum}
%%   아니면 \cftchapnumwidth를 직접 적당히 고치면 된다.
%%%%%%%%%%%%%%%%%%%%%%%%%%%%%%%%%%%%%%%%%%%%%%%%%%%%%%%%%%%%%%%%%%%%%%%%%%%%%%
%\usepackage[titles,subfigure]{tocloft} % when you use subfigure package
\usepackage[titles]{tocloft} % when you don't use subfigure package
\makeatletter
\if@snu@ko
	\renewcommand\cftchappresnum{제~}
	\renewcommand\cftchapaftersnum{~장}
	\renewcommand\cftfigpresnum{그림~}
	\renewcommand\cfttabpresnum{표~}
\else
	\renewcommand\cftchappresnum{Chapter~}
	\renewcommand\cftfigpresnum{Figure~}
	\renewcommand\cfttabpresnum{Table~}
\fi
\makeatother
\newlength{\mytmplen}
\settowidth{\mytmplen}{\bfseries\cftchappresnum\cftchapaftersnum}
\addtolength{\cftchapnumwidth}{\mytmplen}
\settowidth{\mytmplen}{\bfseries\cftfigpresnum\cftfigaftersnum}
\addtolength{\cftfignumwidth}{\mytmplen}
\settowidth{\mytmplen}{\bfseries\cfttabpresnum\cfttabaftersnum}
\addtolength{\cfttabnumwidth}{\mytmplen}
\makeatletter
\g@addto@macro\appendix{%
	\addtocontents{toc}{%
		\settowidth{\mytmplen}{\bfseries\protect\cftchappresnum\protect\cftchapaftersnum}%
		\addtolength{\cftchapnumwidth}{-\mytmplen}%
		\protect\renewcommand{\protect\cftchappresnum}{\appendixname~}%
		\protect\renewcommand{\protect\cftchapaftersnum}{}%
		\settowidth{\mytmplen}{\bfseries\protect\cftchappresnum\protect\cftchapaftersnum}%
		\addtolength{\cftchapnumwidth}{\mytmplen}%
	}%
}
\makeatother
 % SNU toc style

%%%%%%%%%%%%%%%%%%%%%%%%%%%%%%%%%%%%%%%%
%% 다른 패키지 로드
%% http://faq.ktug.or.kr/faq/pdflatex%B0%FAlatex%B5%BF%BD%C3%BB%E7%BF%EB
%% 필요에 따라 직접 수정 필요
\ifpdf
	\input glyphtounicode\pdfgentounicode=1 %type 1 font사용시
	%\usepackage[pdftex,unicode]{hyperref} % delete me
	\usepackage[pdftex]{graphicx}
	%\usepackage[pdftex,svgnames]{xcolor}
\else
	%\usepackage[dvipdfmx,unicode]{hyperref} % delete me
	\usepackage[dvipdfmx]{graphicx}
	%\usepackage[dvipdfmx,svgnames]{xcolor}
\fi
%%%%%%%%%%%%%%%%%%%%%%%%%%%%%%%%%%%%%%%%

\usepackage{lipsum} % lorem ipsum

%% \title : 22pt로 나오는 큰 제목
%% \title* : 16pt로 나오는 작은 제목
\title{한글 제목: 매우 길고 길고 길고\\ %
	길고 길고 길고 길고 긴 부제목}
\title*{English Title: Very\\ Looooooooooooooooooooooooong Subtitle}

\departmentko{전기 컴퓨터 공학부}

%% 저자 이름 Author's(Your) name
\author{홍길동}
\author*{홍~길~동} % Insert space for Hangul name.
%\author{Gildong Hong}
%\author*{Gildong Hong} % Same as \author.

%% 학번 Student number
\studentnumber{2000-00000}

%% 지도교수님 성함 Advisor's name
%% (?) Use Korean name for Korean professor.
\advisor{홍길동}
\advisor*{홍~길~동} % Insert space for Hangul name.
%\advisor{Gildong Hong}
%\advisor*{Gildong Hong}

%% 학위 수여일 Graduation date
%% 표지에 적히는 날짜.
%% 학위 수여일이 아니라 논문 발간년도를 적어야 할 수도 있음.
\graddate{2010~년~2~월}
%\graddate{FEBRUARY 2010}

%% 논문 제출일 Submission date
%% (?) Use Korean date format.
\submissiondate{2009~년~11~월}

%% 논문 인준일 Approval date
%% (?) Use Korean date format.
\approvaldate{2009~년~12~월}

%% 참고: 인준지의 교수님 성함은
%% 컴퓨터로 출력하지 않고, 교수님께서
%% 자필로 쓰시기도 합니다.
%% 위원장, 부위원장, 위원
\committeemembers%
{김교수}%
{이교수}%
{박교수}%
{최교수}%
{John Smith}
%% 밑줄 길이
%\setlength{\committeenameunderlinelength}{7cm}

\begin{document}
\pagenumbering{Roman}
\makefrontcover
\makefrontcover
\makeapproval

\cleardoublepage
\pagenumbering{roman}
% 초록 Abstract
\keyword{서울대학교, 전기공학부, 졸업논문}
\begin{abstract}
서울대학교 전기공학부 졸업논문 예제 파일입니다.
서울대학교 전기공학부 졸업논문 예제 파일입니다.
서울대학교 전기공학부 졸업논문 예제 파일입니다.
서울대학교 전기공학부 졸업논문 예제 파일입니다.
서울대학교 전기공학부 졸업논문 예제 파일입니다.
서울대학교 전기공학부 졸업논문 예제 파일입니다.
서울대학교 전기공학부 졸업논문 예제 파일입니다.
서울대학교 전기공학부 졸업논문 예제 파일입니다.
서울대학교 전기공학부 졸업논문 예제 파일입니다.
서울대학교 전기공학부 졸업논문 예제 파일입니다.

서울대학교 전기공학부 졸업논문 예제 파일입니다.
서울대학교 전기공학부 졸업논문 예제 파일입니다.
서울대학교 전기공학부 졸업논문 예제 파일입니다.
서울대학교 전기공학부 졸업논문 예제 파일입니다.
서울대학교 전기공학부 졸업논문 예제 파일입니다.
서울대학교 전기공학부 졸업논문 예제 파일입니다.
서울대학교 전기공학부 졸업논문 예제 파일입니다.
서울대학교 전기공학부 졸업논문 예제 파일입니다.
서울대학교 전기공학부 졸업논문 예제 파일입니다.
서울대학교 전기공학부 졸업논문 예제 파일입니다.
서울대학교 전기공학부 졸업논문 예제 파일입니다.
서울대학교 전기공학부 졸업논문 예제 파일입니다.
서울대학교 전기공학부 졸업논문 예제 파일입니다.
서울대학교 전기공학부 졸업논문 예제 파일입니다.
서울대학교 전기공학부 졸업논문 예제 파일입니다.
서울대학교 전기공학부 졸업논문 예제 파일입니다.
\end{abstract}

\tableofcontents
\listoftables
\listoffigures

\cleardoublepage
\pagenumbering{arabic}

\chapter{서론}
서론.

\chapter{본론}
본론.

\chapter{결론}
결론.

\appendix

\chapter{내 부록}
\lipsum[1-3]

\begin{thebibliography}{00}
\addcontentsline{toc}{chapter}{\bibname}

% 영문저널의 경우
    \bibitem{ref1} B. Jeon and J. Jeong, ``Blocking artifacts
    reduction in image compression with block boundary discontiunity
    criterion,'' {\em IEEE Transactions on Circuits and Systems for
    Video Tech.}, vol. 8, no.3, pp. 345-357, June 1998.

% 영문학술대회의 경우
    \bibitem{ref2} W. G. Jeon and Y. S. Cho, ``An equalization
    technique for OFDM and MC-CDMA in a multipath fading channels,''
    in {\em Proceedings of IEEE Conference on Acoustics, Speech and
    Signal Processing}, Munich, Germany, May 1997. pp. 2529-2532.

% 국내저널의 경우
    \bibitem{ref3} 김남훈, 정영철, ``평탄한 통과대역 특성을 갖는
    새로운 구조의 광도 파로열 격자 라우터,'' {\em 전자공학회논문지},
    제35권 D편, 제3호, 56-62쪽, 1998년 3월.

% 국내학술대회의 경우
    \bibitem{ref4} 윤남국, 김수종, ``무선 센서 네트워크에서의 에너지
    효율적인 그라디언트 기반 라우팅 기법,'' {\em 한국정보과학회
    2006년 추계학술대회}, 제12권, 제2호, 2006년 10월. pp.
    1372-1374.

% 단행본의 경우
    \bibitem{ref5} C. Mead and L. Conway, {\em Introduction to VLSI
    Systems}, Addison-Wesley, Boston, 1994.

% URL
    \bibitem{ref6} The SolarMESH Network,
    http://owl.mcmater.ca/solarmesh

% Technical Report의 경우
    \bibitem{ref7} K. E. Elliott and C. M. Greene, ``A local adaptive
    protocol,'' Argonne National Laboratory, Argonne, France,
    Technical Report 916-1010-BB, 1997.

% 학위논문의 경우
    \bibitem{ref8} T. Kim, ``Scheduling and Allocation Problems in
    High-level Synthesis,'' Ph. D. Dissertation, ECE Department,
    Univ. of Illinois at U-C, 1993.

% 특허의 경우
    \bibitem{ref9} Sunghyun Choi, ``Wireless MAC protocol based on a
    hybrid combination of slot allocation, token passing, and
    polling for isochronous traffic,'' U.S. Patent No. 6,795,418,
    September 21, 2004.

% 표준
    \bibitem{ref10} IEEE Std. 802.11-1999, Part 11: Wireless LAN
    Medium Access Control (MAC) and Physical Layer (PHY)
    specifications, Reference number ISO/IEC 8802-11:1999(E), IEEE
    Std. 802.11, 1999 edition, 1999.

\end{thebibliography}

\keywordalt{SNU, electrical engineering, thesis}
\begin{abstractalt}
\noindent
\lipsum[1-3]
\end{abstractalt}

\acknowledgement
감사의 글은 생략 가능하고, 형식이 자유이다.

\end{document}

